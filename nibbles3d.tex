\documentclass[10pt]{article}
\usepackage{graphics, verbatim, multicol}
\usepackage{color}
\usepackage{listings}
\usepackage{fullpage}

\definecolor{gray}{rgb}{.4,.4,.4}

\newcommand{\titleize}[1]{
   \begin{center}
       \Large \textsc{#1} \normalsize \\
   \end{center}
}

\newcommand{\normaltitleize}[1]{\mbox{}\\ \textsc{#1} \normalsize}

\newcommand{\headerstuff}{
   \begin{center}
   \textsc{\Large{Alpheus Madsen}}

   \rule{1in}{.01in}

   961 South 200 East\\

   Orem, UT \ \ 84058  \\

   (435)363-5968 \\

   alpheus.madsen@gmail.com

   \rule{2in}{.01in}
   \end{center}
}

\newcommand{\sectionline}{
\begin{center}
   \rule{1in}{.01in}
\end{center}
}

\title{Initial Supplements to A Simple 3D Graphics Engine}
\author{\headerstuff}
\date{}

\begin{document}
\lstset{language=Python, basicstyle=\ttfamily\color{blue}, keywordstyle=\bf\color{blue}, stringstyle=\color{red}, commentstyle=\em\color{gray},
showstringspaces=false,
identifierstyle=\color{black},
numbers=none, numberstyle=\tiny\color{gray}}

%\maketitle
\headerstuff

\titleize{A Simple 3D Graphics Engine\\Written in Python and PyGame}

\begin{center}
   Tuesday, March 3, 2009.
\end{center}

\normaltitleize{Introduction.}  Several years ago I wrote a simple 3D graphics engine in Python and Allegro.  I wanted to grow that into a full-fledged game of some sort, but because of graduate studies, I had to put off working on this little program.  Now that I have completed my doctorate in mathematics, I have spent some time trying to decide what I wanted to do next.  I remembered this program, and how I liked the combination of linear algebra, computer graphics, and pretty pictures, and I decided that I should pursue computer graphics programming.

As I began to look for work, I stumbled onto ``dual quaternions'' and I decided that I wanted to learn more about them.  Thus, I decided to dust off this program, implement dual quaternion math, and use it for a simple game.  In the month that I have done this, I have made much progress; this document is a sampling of the source code of my project.  Although I have corrected many bugs since then, this document only contains source code that represents major changes that I have made.

Unlike my original engine, which used the Allegro library via PyAllegro Python bindings, this new engine uses the SDL game library, through PyGame Python bindings.  Although I liked PyAllegro, the project is currently slow and unstable, and I had problems getting it to work; thus I chose to use a more popular library.  The conversion process went surprisingly well!

\normaltitleize{Progress on My Project.}  When I started my work on this engine, I decided to follow the suggestion of my wife and keep a journal of the progress I have made.  Thus, I will not detail the changes or future goals of my project here; those are provided in a separate document.  I will, however, provide a brief outline of what I have completed:
\begin{itemize}
   \item I wrote wireframe data for worm segments to be used in a Nibbles3D game;

   \item I implemented dual quaternions, and wrote functions that would easily allow for arbitrary combinations of rotations and translations, and convert dual quaternions to matrices;

   \item I implemented a cursor system that would allow me to use the mouse to manipulate the movements of my worms;

   \item I separated wireframe data from position data, to minimize memory used;

   \item I discovered and corrected some subtle camera bugs that, once corrected, allowed things to begin to work in a very intuitive way;

   \item I added coordinate systems to the ``movable'' class, so that I could rotate objects (particularly cameras) by relative coordinates;

   \item I created a ``worm'' class that combines segments and two separate cameras (one for debugging purposes) that will be used for creating worms.
\end{itemize}

\normaltitleize{Personal Remarks.}  As I have revived this project, it has in some ways annoyed me greatly!  Often, I couldn't stop thinking about the things I needed to do, even at times when I so desperately needed to sleep, or other times when I needed to do my work as a computer programmer for EV Source.  I simply wish that I had an infinite amount of time that would allow me to do everything I wanted with this project, and to fulfill all my other obligations to health, family, work, and whatever else I would like to do; at the very least, I wish that my friends and family could somehow box up extra time they had (a few minutes here and there) to send it to me.

On the other hand, working with quaternions, especially dual quaternions, has opened up a new world to me.  Dual quaternions have a simplicity to them that you can't recognize until you really begin to work with them, and to use them in computer graphics especially.  If I do nothing more with my project, this alone was worth reviving it!

%\newpage
\sectionline

\normaltitleize{``quat.py''.}  After converting my program to PyGame, the first thing I did with my project is implement dual quaternions; this was, after all, the reason I was inspired to revive this little program!  I decided that, rather than implement two classes (say, perhaps ``quat'' and ``dualquat''), that I would implement one class, and then test for special cases to determine where I could simplify multiplication or division.

Perhaps it would be more efficient, computationally, to separate things into ``quat'' and ``dualquat'', and then adjust the code accordingly; for now, however, I like this implementation.

\lstinputlisting{quat.py}

% \newpage

\sectionline

\normaltitleize{``position.py''.}  When I started adding segments to my worm, I realized that I was repeatedly reading the same data from the hard drive and then putting it in position; to avoid roundoff error, I would leave the original position information unchanged!  This seemed to be a great waste of both computer memory and computer processes (it takes a lot of work loading files from the disk, just to put things in memory), so I decided to separate the wireframe data from the position data.

\lstinputlisting{position.py}

%\newpage

\sectionline

\normaltitleize{``wf\_data.py''.}  This reads the data that would be used by a ``position'' class; unlike my original ``wireframe'' class, this doesn't inherit from ``movable''; hence, a ``wf\_data'' object would have no position whatsoever.  It's up to ``position'' to place it, and then to draw it when the time comes.

\lstinputlisting{wf_data.py}

% \newpage

\sectionline

\normaltitleize{``worm.py''.}  Since I would like to draw 3D worms that move around gobbling up numbers, I needed a class that would combine the segments and cameras associated with each worm; this class also includes the functions that changes the direction of the worm or its ``fly camera'' (a camera meant to move independently of the worm for debugging purposes).  Since I had to debug the camera, these functions are still very much works in progress.

\lstinputlisting{worm.py}

% \newpage

\sectionline

\normaltitleize{``wormsegments.py''}  The main program that draws and controls the worm.  This file draws the cursor and the ``cursor crosshairs'', and reads movements and clicks from the mouse, (and to a lesser extent, the keyboard), and then moves the camera appropriately.  The worm's ``nose camera'' moves click-by-click (I'm not yet to the point where I want to move forward automatically), while the ``fly camera'' moves in a continuous manner.

\lstinputlisting{latex-wormsegments.py}

% \newpage

\sectionline

\normaltitleize{A note on ``camera.py''.}  Although I have made extensive changes to my camera, I have chosen not to include it here; nonetheless, I wanted to make a brief comment on this file.  The changes to my camera came mostly as a result of discovering that I had reversed my $x$-axis and my $z$-axis in my definitions in ``vertex.py''.  Thus, most of the changes to this class were the result of adjusting constants and removing negative signs.  I also moved Fwd, Left and Up vectors to my ``movable'' class, so that I could use them for keeping track of orientation of objects in general.

I probably still have work to do to (especially with the functions) to make sure I could do everything with my camera that I would hope to do!

%\lstinputlisting{camera.py}
% \newpage

\sectionline

\normaltitleize{``worm\_head\_tail.dat''.}  To implement my worm, I need a head (which could be a tail, too), a middle segment, and an elbow.  This is one of the files I use to create my worm; for incompleteness, I have chosen to leave the other two out of this document.

\lstinputlisting{worm_head_tail.dat}

% \newpage

\end{document}
